\section*{Concluzie}
\phantomsection

Pentru efectuarea acestei lucrări am utilizat un set de instrumente care m-au ajutat la implementarea unei aplicații grafice. Am folosit git ca sistem de control al versiunilor. Acesta oferă posibilitatea de a înregistra fiecare modificare ca o versiune separată precum și crearea mai multor branch-uri. Ca mediu de dezvoltare am folosit IntelliJ care permite redactarea optimă a codului Java și are integrarea cu git și Maven. Pentru a controla structura și modul de construire a proiectului, precum și includerea eficientă și rapidă a dependențelor externe am folosit managerul Maven. Una din dependențele principale utilizate este biblioteca gratuita create de Apache foundation numită Pivot care oferă posibilitatea de a construi interfețe grafice. Ea permite definirea înfățișării folosind un fișier XML apoi manipularea acestor componente și atașarea evenimentelor folosind anotarea @BXML. Toate aceste instrumente mi-au permis sa dezvolt eficient un produs software calitativ.

\clearpage